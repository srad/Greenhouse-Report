\graphicspath{{members/ssr/figures/modelling}}

\subsection{Design Overview}\label{null:design-overview}
\input{members/ssr/authors}

The designed system for the purpose of yield prediction is founded on one central assumption: 
the main metric for vital growth is the leaf area index (LAI) which determines the PAR:

\begin{quote}
    Leaf area index (LAI) is the total one‐sided area of leaf tissue per unit ground surface area.
    It is a key parameter in ecophysiology, especially for scaling up the gas exchange from leaf
    to canopy level.
    It characterizes the canopy–atmosphere interface, where most of the energy fluxes exchange. \cite{beda:nathalie}
\end{quote}

For this purpose a plant model is defined in the following section which is used to estimate
the LAI.

\graphicspath{{members/ssr/figures/}}

\subsubsection*{Plant Model}

Based on the physiological description of tomato plants -- as provided by the user view -- a plant model has
been chosen as shown by Figure \ref{fig:plant:model:2}.

\begin{figure}[H]
    \centering
    \includegraphics[width=0.9\textwidth,height=\textheight,keepaspectratio]{modelling/plant-model-3.png}
    \caption{Tomato plant model}
    \label{fig:plant:model:2}
\end{figure}

This model especially serves for the purpose of LAI calculation where the canopy visible from above can 
be mapped to the largest (and lowest) branch level of the plant -- whereby each higher (and therefore newer)
level is smaller.
The size of the plant's entire leaf area is then correlated with the number of levels.
This model will be formalized in more detail later with all of its parameters. 

\subsection{Computational Pipelines}\label{subsec:computational-pipelines}
\input{members/ssr/authors}

To estimate the central LAI metric, two computation pipelines are combined to extract specific metrics from images.
Each pipeline applies a series of image processing and computations in order to contribute to the result
in the following ways:

\begin{enumerate}
    \item \textbf{Pipeline A:} Estimates the \textit{leaf area} (not LAI) which is the visible canopy
    of a single plant per ground unit, captured by camera mounted above of the plants (see Figure \ref{fig:pipeline:a:camera:setup})
    \item \textbf{Pipeline B:} Estimates the plant height, based on an image taken from a different camera and position,
    with a specific setup.
\end{enumerate}

These both metrics are then combined (with additional statistical corrections) to estimate the actual LAI.
Again, the LAI is different from the \textit{leaf area} in the way that it contains
the \textit{entire} existing leaf area of a plant and not only the leaf area which is visible from above.
it's a common issue to map from a captured imaged from above to the actual plant's leaf area in order
to estimate the plant's bio-activity.

\subsection{Pipeline A}\label{subsec:pipeline-a}

The purpose of Pipeline A is to compute the leaf area taken from an image from above.
In order to do that based on an captured photo image a series of image processing and
statistical correction has to applied. Figure \ref{fig:p:a} illustrates the high level
view of the pipeline in order to achieve this. Each step is presented in the following sections.

\begin{figure}[H]
    \centering
    \includegraphics[width=1.0\textwidth]{modelling/pipeline-a.png}
    \caption{The pipeline can be accessed by web browser or purely on a Windows desktop.}
    \label{fig:p:a}
\end{figure}

Since the first step of the computational pipeline is an input image the following section
will describe how the image is acquired.

\input{members/ssr/modelling/setup.tex}
\input{members/ssr/modelling/clustering.tex}
